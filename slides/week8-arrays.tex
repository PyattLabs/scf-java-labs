\documentclass{beamer}
\usetheme{Madrid}
\usecolortheme{dolphin}
\definecolor{charcoal}{RGB}{54,69,79}
\setbeamercolor{structure}{fg=charcoal}
\setbeamercolor{palette primary}{bg=charcoal,fg=white}
\setbeamercolor{palette secondary}{bg=charcoal!90,fg=white}
\setbeamercolor{palette tertiary}{bg=charcoal!80,fg=white}
\setbeamercolor{palette quaternary}{bg=charcoal!70,fg=white}
\setbeamercolor{frametitle}{bg=charcoal,fg=white}
\setbeamercolor{title}{bg=charcoal,fg=white}

\usepackage{hyperref}
\usepackage{graphicx}
\usepackage{booktabs}
\usepackage{listings}
\usepackage{color}

\definecolor{dkgreen}{rgb}{0,0.6,0}
\definecolor{gray}{rgb}{0.5,0.5,0.5}
\definecolor{mauve}{rgb}{0.58,0,0.82}

\lstset{
  language=Java,
  aboveskip=3mm,
  belowskip=3mm,
  showstringspaces=false,
  columns=flexible,
  basicstyle={\small\ttfamily},
  numbers=none,
  keywordstyle=\color{blue},
  commentstyle=\color{dkgreen},
  stringstyle=\color{mauve},
  breaklines=true,
  breakatwhitespace=true,
  tabsize=2
}

\title[Single-Dim Arrays]{Single-Dimensional Arrays}
\subtitle{COP2250: Java Programming}
\author{Kevin Pyatt, Ph.D.}
\institute{State College of Florida \\ Pyatt Labs}
\date{Week 8}

\begin{document}

% TITLE SLIDE
\begin{frame}
  \titlepage
\end{frame}

% OBJECTIVES
\begin{frame}{Today's Objectives}
\begin{itemize}
  \item Declare and create arrays
  \item Access elements with index notation
  \item Process arrays with loops
  \item Find max, min, sum, and average of array elements
  \item Understand array bounds and \texttt{ArrayIndexOutOfBoundsException}
  \item Use \texttt{length} property
\end{itemize}
\end{frame}

% THE PROBLEM
\section{Why Arrays?}
\begin{frame}[fragile]{The Problem: Too Many Variables}
\textbf{Store 100 student scores without arrays:}
\begin{lstlisting}
double score1 = 85.0;
double score2 = 92.0;
double score3 = 78.0;
// ... 97 more ...
\end{lstlisting}

\vspace{0.5cm}
\textbf{With an array:}
\begin{lstlisting}
double[] scores = new double[100];
\end{lstlisting}

\vspace{0.3cm}
\textit{One variable, 100 values. Arrays let you store collections of the same type.}
\end{frame}

% DECLARING AND CREATING
\section{Array Basics}
\begin{frame}[fragile]{Declaring and Creating Arrays}
\textbf{Two steps:}
\begin{lstlisting}
// Declare and create (empty, default values)
double[] scores = new double[5];

// Declare and initialize with values
int[] nums = {10, 20, 30, 40, 50};
\end{lstlisting}

\vspace{0.3cm}
\begin{tabular}{ll}
\texttt{double[]} & Type is ``array of doubles'' \\
\texttt{new double[5]} & Creates 5 slots, all initialized to 0.0 \\
\texttt{\{10, 20, 30\}} & Shorthand --- size determined by values \\
\end{tabular}

\vspace{0.3cm}
\textbf{Default values:} \texttt{int} = 0, \texttt{double} = 0.0, \texttt{boolean} = false, \texttt{String} = null
\end{frame}

% INDEXING
\begin{frame}[fragile]{Accessing Elements}
\begin{lstlisting}
int[] nums = {10, 20, 30, 40, 50};

System.out.println(nums[0]);  // 10 (first)
System.out.println(nums[4]);  // 50 (last)
System.out.println(nums[5]);  // CRASH!

nums[2] = 99;  // Change third element
\end{lstlisting}

\vspace{0.3cm}
\begin{itemize}
  \item Index starts at \textbf{0}, ends at \textbf{length - 1}
  \item \texttt{nums[5]} on a size-5 array $\rightarrow$ \texttt{ArrayIndexOutOfBoundsException}
  \item \texttt{nums.length} returns 5 (not an index!)
\end{itemize}

\vspace{0.3cm}
\textbf{Mental model:} 5 boxes numbered 0 through 4. No box 5.
\end{frame}

% PROCESSING WITH LOOPS
\section{Processing Arrays}
\begin{frame}[fragile]{Processing Arrays with Loops}
\textbf{Print all elements:}
\begin{lstlisting}
for (int i = 0; i < scores.length; i++) {
    System.out.println("Score " + i + ": " + scores[i]);
}
\end{lstlisting}

\vspace{0.3cm}
\textbf{Sum all elements:}
\begin{lstlisting}
double total = 0;
for (int i = 0; i < scores.length; i++) {
    total += scores[i];
}
double average = total / scores.length;
\end{lstlisting}

\vspace{0.3cm}
\textbf{Pattern:} \texttt{for (int i = 0; i < array.length; i++)} \\
You will write this loop hundreds of times. Memorize it.
\end{frame}

% FINDING MAX
\begin{frame}[fragile]{Finding the Maximum}
\begin{lstlisting}
double max = scores[0];  // Start with first

for (int i = 1; i < scores.length; i++) {
    if (scores[i] > max) {
        max = scores[i];
    }
}

System.out.println("Best score: " + max);
\end{lstlisting}

\vspace{0.3cm}
\textbf{Key idea:}
\begin{itemize}
  \item Assume the first element is the max
  \item Walk through the rest --- if you find bigger, update
  \item Loop starts at \texttt{i = 1} (already checked index 0)
\end{itemize}

\vspace{0.2cm}
\textit{Same pattern works for min --- just flip the comparison.}
\end{frame}

% ENHANCED FOR LOOP
\begin{frame}[fragile]{Enhanced For Loop (for-each)}
\begin{lstlisting}
// Standard for loop
for (int i = 0; i < scores.length; i++) {
    System.out.println(scores[i]);
}

// Enhanced for loop (same result)
for (double s : scores) {
    System.out.println(s);
}
\end{lstlisting}

\vspace{0.3cm}
\textbf{Use enhanced for when:}
\begin{itemize}
  \item You need every element
  \item You do NOT need the index
\end{itemize}

\textbf{Use standard for when:}
\begin{itemize}
  \item You need the index
  \item You need to modify elements
  \item You need to go backwards or skip elements
\end{itemize}
\end{frame}

% INPUT INTO ARRAYS
\begin{frame}[fragile]{Reading Input into an Array}
\begin{lstlisting}
Scanner input = new Scanner(System.in);

System.out.print("How many scores? ");
int n = input.nextInt();

double[] scores = new double[n];

for (int i = 0; i < n; i++) {
    System.out.print("Score " + (i + 1) + ": ");
    scores[i] = input.nextDouble();
}
\end{lstlisting}

\vspace{0.3cm}
\textbf{Pattern:}
\begin{enumerate}
  \item Ask how many
  \item Create array of that size
  \item Loop to fill it
\end{enumerate}
\end{frame}

% COMMON MISTAKES
\begin{frame}[fragile]{Common Mistakes}
\textbf{1. Off-by-one:}
\begin{lstlisting}
for (int i = 0; i <= scores.length; i++) // CRASH
for (int i = 0; i < scores.length; i++)  // CORRECT
\end{lstlisting}

\vspace{0.2cm}
\textbf{2. Uninitialized array:}
\begin{lstlisting}
double[] scores;       // declared, not created
scores[0] = 85;        // NullPointerException
\end{lstlisting}

\vspace{0.2cm}
\textbf{3. Confusing length with last index:}
\begin{lstlisting}
// Array of 5: indices 0-4, length is 5
scores[scores.length]     // CRASH (index 5)
scores[scores.length - 1] // last element
\end{lstlisting}

\vspace{0.2cm}
\textbf{4. Wrong initial max:}
\begin{lstlisting}
double max = 0;         // WRONG if all scores negative
double max = scores[0]; // CORRECT
\end{lstlisting}
\end{frame}

% WRAP UP
\begin{frame}{Assignment 7: Grade by Curve}
\textbf{Exercise 7.1:} Write a program that:
\begin{itemize}
  \item Reads student scores into an array
  \item Finds the best score
  \item Assigns grades on a curve:
\end{itemize}

\begin{tabular}{ll}
\textbf{Grade A} & score $\geq$ best $-$ 10 \\
\textbf{Grade B} & score $\geq$ best $-$ 20 \\
\textbf{Grade C} & score $\geq$ best $-$ 30 \\
\textbf{Grade D} & score $\geq$ best $-$ 40 \\
\textbf{Grade F} & otherwise \\
\end{tabular}

\vspace{0.5cm}
\textbf{Lab: ArrayPractice}
\begin{itemize}
  \item Build array utility methods step by step
  \item Sum, average, max, min, count, search
\end{itemize}

\vspace{0.3cm}
\textbf{Next Week:} Multi-Dimensional Arrays (Chapter 8)
\end{frame}

\end{document}
