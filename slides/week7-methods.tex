\documentclass{beamer}
\usetheme{Madrid}
\usecolortheme{dolphin}
\definecolor{charcoal}{RGB}{54,69,79}
\setbeamercolor{structure}{fg=charcoal}
\setbeamercolor{palette primary}{bg=charcoal,fg=white}
\setbeamercolor{palette secondary}{bg=charcoal!90,fg=white}
\setbeamercolor{palette tertiary}{bg=charcoal!80,fg=white}
\setbeamercolor{palette quaternary}{bg=charcoal!70,fg=white}
\setbeamercolor{frametitle}{bg=charcoal,fg=white}
\setbeamercolor{title}{bg=charcoal,fg=white}

\usepackage{hyperref}
\usepackage{graphicx}
\usepackage{booktabs}
\usepackage{listings}
\usepackage{color}

\definecolor{dkgreen}{rgb}{0,0.6,0}
\definecolor{gray}{rgb}{0.5,0.5,0.5}
\definecolor{mauve}{rgb}{0.58,0,0.82}

\lstset{
  language=Java,
  aboveskip=3mm,
  belowskip=3mm,
  showstringspaces=false,
  columns=flexible,
  basicstyle={\small\ttfamily},
  numbers=none,
  keywordstyle=\color{blue},
  commentstyle=\color{dkgreen},
  stringstyle=\color{mauve},
  breaklines=true,
  breakatwhitespace=true,
  tabsize=2
}

\title[Methods]{Methods}
\subtitle{COP2250: Java Programming}
\author{Kevin Pyatt, Ph.D.}
\institute{State College of Florida \\ Pyatt Labs}
\date{Week 7}

\begin{document}

% TITLE SLIDE
\begin{frame}
  \titlepage
\end{frame}

% OBJECTIVES
\begin{frame}{Today's Objectives}
\begin{itemize}
  \item Understand what methods are and why we use them
  \item Define methods with parameters and return values
  \item Distinguish \texttt{void} methods from value-returning methods
  \item Call methods and pass arguments
  \item Understand method signatures and overloading
  \item Apply the \textbf{divide and conquer} principle
\end{itemize}
\end{frame}

% THE PROBLEM
\section{Why Methods?}
\begin{frame}[fragile]{The Problem: Copy-Paste Code}
\textbf{What if you need the same logic in 5 places?}
\begin{lstlisting}
// Find max of two numbers... again
int max1;
if (a > b) max1 = a; else max1 = b;

// Same logic, different variables
int max2;
if (x > y) max2 = x; else max2 = y;
\end{lstlisting}
\vspace{0.3cm}
\textbf{Problems:}
\begin{itemize}
  \item Duplicate code everywhere
  \item Fix a bug? Fix it in 5 places
  \item Hard to read, hard to maintain
\end{itemize}
\vspace{0.3cm}
\textbf{Methods let you write it once, call it anywhere.}
\end{frame}

% ANATOMY OF A METHOD
\section{Defining Methods}
\begin{frame}[fragile]{Anatomy of a Method}
\begin{lstlisting}
public static int max(int num1, int num2) {
    if (num1 > num2) {
        return num1;
    } else {
        return num2;
    }
}
\end{lstlisting}

\begin{table}
\centering
\begin{tabular}{ll}
\toprule
\textbf{Part} & \textbf{Meaning} \\
\midrule
\texttt{public static} & Modifiers (we'll use these for now) \\
\texttt{int} & Return type \\
\texttt{max} & Method name \\
\texttt{(int num1, int num2)} & Parameters \\
\texttt{return num1;} & Return statement \\
\bottomrule
\end{tabular}
\end{table}
\end{frame}

% VOID METHODS
\begin{frame}[fragile]{\texttt{void} Methods}
A method that does something but \textbf{returns nothing}.

\begin{lstlisting}
public static void printGreeting(String name) {
    System.out.println("Hello, " + name + "!");
}
\end{lstlisting}

\textbf{Calling it:}
\begin{lstlisting}
printGreeting("Kevin");  // prints: Hello, Kevin!
\end{lstlisting}

\begin{itemize}
  \item Return type is \texttt{void} --- no return value
  \item You \textbf{call} it as a statement, not in an expression
  \item Used for: printing, display, side effects
\end{itemize}
\end{frame}

% VALUE-RETURNING METHODS
\begin{frame}[fragile]{Value-Returning Methods}
A method that computes and \textbf{sends back} a result.

\begin{lstlisting}
public static double celsiusToFahrenheit(double c) {
    return (9.0 / 5) * c + 32;
}
\end{lstlisting}

\textbf{Calling it:}
\begin{lstlisting}
double temp = celsiusToFahrenheit(100);
System.out.println(temp);  // 212.0
\end{lstlisting}

\begin{itemize}
  \item Return type matches what \texttt{return} sends back
  \item You \textbf{use} the result --- assign it, print it, pass it
  \item Every path through the method must hit a \texttt{return}
\end{itemize}
\end{frame}

% CALLING METHODS
\section{Calling Methods}
\begin{frame}[fragile]{Arguments vs Parameters}
\begin{lstlisting}
// DEFINITION: num1 and num2 are PARAMETERS
public static int max(int num1, int num2) {
    return (num1 > num2) ? num1 : num2;
}

// CALL: 5 and 3 are ARGUMENTS
int result = max(5, 3);
\end{lstlisting}

\begin{itemize}
  \item \textbf{Parameters}: variables in the method header (placeholders)
  \item \textbf{Arguments}: actual values you pass when calling
  \item Arguments must match parameters in \textbf{order, number, and compatible type}
\end{itemize}

\vspace{0.3cm}
\textbf{Think of it like a function in math:} $f(x) = x^2$ \\
$x$ is the parameter. $f(3) = 9$ --- 3 is the argument.
\end{frame}

% METHOD OVERLOADING
\begin{frame}[fragile]{Method Overloading}
Same name, \textbf{different parameters}.

\begin{lstlisting}
public static int max(int a, int b) {
    return (a > b) ? a : b;
}

public static double max(double a, double b) {
    return (a > b) ? a : b;
}

public static int max(int a, int b, int c) {
    return max(max(a, b), c);
}
\end{lstlisting}

\begin{itemize}
  \item Java picks the right version based on arguments
  \item Overloading = same name, different \textbf{signature}
  \item Signature = method name + parameter types
  \item Return type alone does NOT distinguish methods
\end{itemize}
\end{frame}

% SCOPE
\begin{frame}[fragile]{Variable Scope}
Variables declared inside a method are \textbf{local} to that method.

\begin{lstlisting}
public static void methodA() {
    int x = 10;  // x lives here only
}

public static void methodB() {
    System.out.println(x);  // ERROR: x not found
}
\end{lstlisting}

\begin{itemize}
  \item Local variables are created when the method is called
  \item They are destroyed when the method returns
  \item Two methods can have variables with the same name --- they are \textbf{different variables}
  \item Pass data between methods via \textbf{parameters and return values}
\end{itemize}
\end{frame}

% COMMON MISTAKES
\begin{frame}[fragile]{Common Mistakes}
\textbf{1. Forgetting \texttt{return}:}
\begin{lstlisting}
public static int square(int n) {
    int result = n * n;
    // missing return result;
}
\end{lstlisting}

\textbf{2. Calling void method in expression:}
\begin{lstlisting}
int x = printGreeting("Kevin"); // ERROR
\end{lstlisting}

\textbf{3. Wrong argument types:}
\begin{lstlisting}
max("hello", "world"); // expects int, not String
\end{lstlisting}

\textbf{4. Defining a method inside another method:}
\begin{lstlisting}
public static void main(String[] args) {
    public static void broken() { } // ERROR
}
\end{lstlisting}
\end{frame}

% TODAY'S LAB
\begin{frame}{Today's Lab + Walkthrough}
\textbf{Method Practice Lab}
\begin{itemize}
  \item Write a \texttt{void} method that prints a grade from a score
  \item Write a value-returning method that finds max of three numbers
  \item Write a method that checks if a number is even
  \item Call each method from \texttt{main} with user input
\end{itemize}
\vspace{0.3cm}
\textbf{Assignment 6:} Write \texttt{displaySortedNumbers(double, double, double)} --- sort and display three numbers in increasing order.
\vspace{0.3cm}

\textbf{Last 15 minutes: Code Walkthrough} \\
\begin{itemize}
  \item Screen-share your code
  \item Walk me through it line by line
  \item Explain \textit{why}, not just \textit{what}
  \item No working code = no credit
\end{itemize}
\end{frame}

\end{document}
