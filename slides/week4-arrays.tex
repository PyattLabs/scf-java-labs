\documentclass{beamer}
\usetheme{Madrid}
\usecolortheme{dolphin}
\definecolor{charcoal}{RGB}{54,69,79}
\setbeamercolor{structure}{fg=charcoal}
\setbeamercolor{palette primary}{bg=charcoal,fg=white}
\setbeamercolor{palette secondary}{bg=charcoal!90,fg=white}
\setbeamercolor{palette tertiary}{bg=charcoal!80,fg=white}
\setbeamercolor{palette quaternary}{bg=charcoal!70,fg=white}
\setbeamercolor{frametitle}{bg=charcoal,fg=white}
\setbeamercolor{title}{bg=charcoal,fg=white}

\usepackage{hyperref}
\usepackage{graphicx}
\usepackage{booktabs}
\usepackage{listings}
\usepackage{color}

\definecolor{dkgreen}{rgb}{0,0.6,0}
\definecolor{gray}{rgb}{0.5,0.5,0.5}
\definecolor{mauve}{rgb}{0.58,0,0.82}

\lstset{
  language=Java,
  aboveskip=3mm,
  belowskip=3mm,
  showstringspaces=false,
  columns=flexible,
  basicstyle={\small\ttfamily},
  numbers=none,
  keywordstyle=\color{blue},
  commentstyle=\color{dkgreen},
  stringstyle=\color{mauve},
  breaklines=true,
  breakatwhitespace=true,
  tabsize=2
}

\title[Arrays]{Arrays}
\subtitle{COP2250: Java Programming}
\author{Kevin Pyatt, Ph.D.}
\institute{State College of Florida \\ Pyatt Labs}
\date{Week 4}

\begin{document}

% TITLE SLIDE
\begin{frame}
  \titlepage
\end{frame}

% OBJECTIVES
\begin{frame}{Today's Objectives}
\begin{itemize}
  \item Understand what arrays are and why we use them
  \item Declare and initialize arrays
  \item Access elements by index (zero-indexed)
  \item Use \texttt{.length} property
  \item Loop through arrays
  \item Apply arrays to Rock-Paper-Scissors
\end{itemize}
\end{frame}

% THE PROBLEM
\section{The Problem}
\begin{frame}[fragile]{The Problem: Too Many Variables}
\textbf{What if you need 100 test scores?}
\begin{lstlisting}
int score1 = 90;
int score2 = 85;
int score3 = 78;
// ... 97 more lines?!
\end{lstlisting}

\vspace{0.5cm}
\textbf{This doesn't scale.}
\begin{itemize}
  \item Tedious to write
  \item Hard to maintain
  \item Can't loop through individual variables
\end{itemize}
\end{frame}

% THE SOLUTION
\begin{frame}[fragile]{The Solution: Arrays}
\textbf{An array holds multiple values in one variable.}

\begin{lstlisting}
int[] scores = {90, 85, 78, 92, 88};
\end{lstlisting}

\vspace{0.5cm}
\textbf{Think of it as a row of boxes:}
\begin{center}
\begin{tabular}{|c|c|c|c|c|}
\hline
90 & 85 & 78 & 92 & 88 \\
\hline
\texttt{[0]} & \texttt{[1]} & \texttt{[2]} & \texttt{[3]} & \texttt{[4]} \\
\hline
\end{tabular}
\end{center}

\vspace{0.3cm}
Each box has an \textbf{index} starting at \textbf{0}.
\end{frame}

% DECLARING ARRAYS
\section{Declaring Arrays}
\begin{frame}[fragile]{Declaring Arrays}
\textbf{Two ways to create an array:}

\vspace{0.5cm}
\textbf{1. Declare with initial values:}
\begin{lstlisting}
int[] scores = {90, 85, 78, 92, 88};
String[] names = {"Alice", "Bob", "Charlie"};
\end{lstlisting}

\vspace{0.5cm}
\textbf{2. Declare empty, fill later:}
\begin{lstlisting}
String[] colors = new String[3];  // 3 empty slots
colors[0] = "Red";
colors[1] = "Green";
colors[2] = "Blue";
\end{lstlisting}
\end{frame}

% ACCESSING ELEMENTS
\section{Accessing Elements}
\begin{frame}[fragile]{Accessing Elements}
\textbf{Arrays are zero-indexed --- first element is at index 0.}

\begin{lstlisting}
String[] choices = {"Scissor", "Rock", "Paper"};

System.out.println(choices[0]);  // Scissor
System.out.println(choices[1]);  // Rock
System.out.println(choices[2]);  // Paper
\end{lstlisting}

\vspace{0.5cm}
\begin{center}
\begin{tabular}{|c|c|c|}
\hline
"Scissor" & "Rock" & "Paper" \\
\hline
\texttt{[0]} & \texttt{[1]} & \texttt{[2]} \\
\hline
\end{tabular}
\end{center}
\end{frame}

% COMMON MISTAKE
\begin{frame}[fragile]{Common Mistake: Off-By-One}
\textbf{Arrays start at 0, not 1!}

\begin{lstlisting}
String[] choices = {"Scissor", "Rock", "Paper"};

// WRONG - will crash!
System.out.println(choices[3]);  
// ArrayIndexOutOfBoundsException

// RIGHT - last index is length - 1
System.out.println(choices[2]);  // Paper
\end{lstlisting}

\vspace{0.3cm}
\textbf{Remember:} For an array of size $n$, valid indices are $0$ to $n-1$.
\end{frame}

% ARRAY LENGTH
\section{Array Length}
\begin{frame}[fragile]{Array Length}
\textbf{Use \texttt{.length} to get the size of an array.}

\begin{lstlisting}
String[] choices = {"Scissor", "Rock", "Paper"};

System.out.println(choices.length);  // 3
\end{lstlisting}

\vspace{0.5cm}
\textbf{Useful for:}
\begin{itemize}
  \item Looping through all elements
  \item Finding the last element: \texttt{array[array.length - 1]}
  \item Avoiding out-of-bounds errors
\end{itemize}

\vspace{0.3cm}
\textbf{Note:} It's \texttt{.length} (no parentheses), not \texttt{.length()}
\end{frame}

% LOOPING
\section{Looping Through Arrays}
\begin{frame}[fragile]{Looping Through Arrays}
\textbf{Use a for loop to visit every element:}

\begin{lstlisting}
int[] scores = {90, 85, 78, 92, 88};

for (int i = 0; i < scores.length; i++) {
    System.out.println("Score " + i + ": " + scores[i]);
}
\end{lstlisting}

\vspace{0.3cm}
\textbf{Output:}
\begin{verbatim}
Score 0: 90
Score 1: 85
Score 2: 78
Score 3: 92
Score 4: 88
\end{verbatim}
\end{frame}

% CALCULATE AVERAGE
\begin{frame}[fragile]{Example: Calculate Average}
\begin{lstlisting}
int[] scores = {90, 85, 78, 92, 88};
int sum = 0;

for (int i = 0; i < scores.length; i++) {
    sum = sum + scores[i];
}

double average = (double) sum / scores.length;
System.out.println("Average: " + average);
\end{lstlisting}

\vspace{0.3cm}
\textbf{Output:} \texttt{Average: 86.6}

\vspace{0.3cm}
\textbf{Note:} Cast to \texttt{double} to avoid integer division.
\end{frame}

% ROCK PAPER SCISSORS
\section{Application: Rock-Paper-Scissors}
\begin{frame}[fragile]{Rock-Paper-Scissors with Arrays}
\textbf{Store choices in an array:}

\begin{lstlisting}
String[] choices = {"Scissor", "Rock", "Paper"};
//                     0          1        2
\end{lstlisting}

\vspace{0.3cm}
\textbf{Computer picks a random number:}
\begin{lstlisting}
Random rand = new Random();
int computer = rand.nextInt(3);  // 0, 1, or 2

System.out.println("Computer chose: " + choices[computer]);
\end{lstlisting}

\vspace{0.3cm}
\textbf{User enters a number, we use it as index:}
\begin{lstlisting}
int user = input.nextInt();
System.out.println("You chose: " + choices[user]);
\end{lstlisting}
\end{frame}

% RPS LOGIC
\begin{frame}{Rock-Paper-Scissors: Game Logic}
\textbf{Index mapping:}
\begin{center}
\begin{tabular}{|c|c|}
\hline
\textbf{Index} & \textbf{Choice} \\
\hline
0 & Scissor \\
1 & Rock \\
2 & Paper \\
\hline
\end{tabular}
\end{center}

\vspace{0.5cm}
\textbf{User wins when:}
\begin{itemize}
  \item user == 0 AND computer == 2 (scissor cuts paper)
  \item user == 1 AND computer == 0 (rock smashes scissor)
  \item user == 2 AND computer == 1 (paper wraps rock)
\end{itemize}
\end{frame}

% RANDOM CLASS
\begin{frame}[fragile]{The Random Class}
\textbf{Import and create:}
\begin{lstlisting}
import java.util.Random;

Random rand = new Random();
\end{lstlisting}

\vspace{0.5cm}
\textbf{Generate random integers:}
\begin{lstlisting}
int n = rand.nextInt(3);  // Returns 0, 1, or 2
int m = rand.nextInt(10); // Returns 0 through 9
int p = rand.nextInt(6) + 1; // Returns 1 through 6 (dice)
\end{lstlisting}

\vspace{0.3cm}
\textbf{Pattern:} \texttt{nextInt(max)} returns 0 to max-1
\end{frame}

% SUMMARY
\section{Summary}
\begin{frame}{Summary}
\begin{itemize}
  \item \textbf{Arrays} store multiple values of the same type
  \item \textbf{Zero-indexed} --- first element is at index 0
  \item \textbf{Declare:} \texttt{int[] arr = \{1, 2, 3\};}
  \item \textbf{Access:} \texttt{arr[0]}, \texttt{arr[1]}, etc.
  \item \textbf{Length:} \texttt{arr.length} (no parentheses)
  \item \textbf{Loop:} \texttt{for (int i = 0; i < arr.length; i++)}
  \item \textbf{Random:} \texttt{rand.nextInt(n)} gives 0 to n-1
\end{itemize}
\end{frame}

% LAB PREVIEW
\begin{frame}{Lab 2: Array Practice}
\textbf{Complete \texttt{ArrayPractice.java}:}
\begin{enumerate}
  \item Create integer array with test scores
  \item Print first and last elements
  \item Calculate average using a loop
  \item Create String array for colors
  \item Print each color with a loop
  \item Create Rock-Paper-Scissors choices array
\end{enumerate}

\vspace{0.5cm}
\textbf{This prepares you for Assignment 3!}
\end{frame}

% ASSIGNMENT PREVIEW
\begin{frame}{Assignment 3: Rock-Paper-Scissors}
\textbf{Build a working game:}
\begin{enumerate}
  \item Create choices array
  \item Generate random computer choice (0-2)
  \item Get user input (0-2)
  \item Display both choices using the array
  \item Determine winner with if/else
\end{enumerate}

\vspace{0.5cm}
\textbf{Starter code and reference sheet in the repo!}
\end{frame}

% END
\begin{frame}
\begin{center}
\Huge Questions?

\vspace{1cm}
\large
Lab 2: \texttt{ArrayPractice.java}

\vspace{0.5cm}
Assignment 3: Rock-Paper-Scissors
\end{center}
\end{frame}

\end{document}
