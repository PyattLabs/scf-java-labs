\documentclass{beamer}
\usetheme{Madrid}
\usecolortheme{dolphin}
\definecolor{charcoal}{RGB}{54,69,79}
\setbeamercolor{structure}{fg=charcoal}
\setbeamercolor{palette primary}{bg=charcoal,fg=white}
\setbeamercolor{palette secondary}{bg=charcoal!90,fg=white}
\setbeamercolor{palette tertiary}{bg=charcoal!80,fg=white}
\setbeamercolor{palette quaternary}{bg=charcoal!70,fg=white}
\setbeamercolor{frametitle}{bg=charcoal,fg=white}
\setbeamercolor{title}{bg=charcoal,fg=white}

\usepackage{hyperref}
\usepackage{graphicx}
\usepackage{booktabs}
\usepackage{listings}
\usepackage{color}

\definecolor{dkgreen}{rgb}{0,0.6,0}
\definecolor{gray}{rgb}{0.5,0.5,0.5}
\definecolor{mauve}{rgb}{0.58,0,0.82}

\lstset{
  language=Java,
  aboveskip=3mm,
  belowskip=3mm,
  showstringspaces=false,
  columns=flexible,
  basicstyle={\small\ttfamily},
  numbers=none,
  keywordstyle=\color{blue},
  commentstyle=\color{dkgreen},
  stringstyle=\color{mauve},
  breaklines=true,
  breakatwhitespace=true,
  tabsize=2
}

\title[Loops]{Loops}
\subtitle{COP2250: Java Programming}
\author{Kevin Pyatt, Ph.D.}
\institute{State College of Florida \\ Pyatt Labs}
\date{Week 6}

\begin{document}

% TITLE SLIDE
\begin{frame}
  \titlepage
\end{frame}

% OBJECTIVES
\begin{frame}{Today's Objectives}
\begin{itemize}
  \item Understand the three loop types: \texttt{while}, \texttt{do-while}, \texttt{for}
  \item Recognize when to use each loop
  \item Use sentinel values to control input
  \item Accumulate totals and counts inside loops
  \item Avoid infinite loops and off-by-one errors
\end{itemize}
\end{frame}

% THE PROBLEM
\section{Why Loops?}
\begin{frame}[fragile]{The Problem: Repetition Without Loops}
\textbf{Print ``Hello'' 5 times without a loop:}
\begin{lstlisting}
System.out.println("Hello");
System.out.println("Hello");
System.out.println("Hello");
System.out.println("Hello");
System.out.println("Hello");
\end{lstlisting}
\vspace{0.5cm}
Now do it 1,000 times?

\textbf{Loops let you repeat code without copying it.}
\end{frame}

% WHILE LOOP
\section{while Loop}
\begin{frame}[fragile]{The \texttt{while} Loop}
\textbf{Syntax:}
\begin{lstlisting}
while (condition) {
    // body executes while condition is true
}
\end{lstlisting}

\textbf{Example --- count from 1 to 5:}
\begin{lstlisting}
int count = 1;
while (count <= 5) {
    System.out.println(count);
    count++;
}
\end{lstlisting}

\begin{itemize}
  \item Checks condition \textbf{before} each iteration
  \item Body may execute \textbf{zero} times
  \item Must update the loop variable or you get an infinite loop
\end{itemize}
\end{frame}

% SENTINEL VALUE
\begin{frame}[fragile]{Sentinel Values}
A \textbf{sentinel value} signals ``stop reading input.''

\begin{lstlisting}
System.out.print("Enter a number (0 to quit): ");
int num = input.nextInt();

while (num != 0) {
    // process num
    System.out.print("Enter a number (0 to quit): ");
    num = input.nextInt();
}
\end{lstlisting}

\begin{itemize}
  \item The sentinel (0) is \textbf{not} processed as data
  \item Read \textbf{before} the loop, then again at the \textbf{end} of the loop body
  \item This is called a \textbf{priming read}
\end{itemize}
\end{frame}

% DO-WHILE LOOP
\section{do-while Loop}
\begin{frame}[fragile]{The \texttt{do-while} Loop}
\textbf{Syntax:}
\begin{lstlisting}
do {
    // body executes at least once
} while (condition);
\end{lstlisting}

\textbf{Example --- input validation:}
\begin{lstlisting}
int choice;
do {
    System.out.print("Enter 1, 2, or 3: ");
    choice = input.nextInt();
} while (choice < 1 || choice > 3);
\end{lstlisting}

\begin{itemize}
  \item Checks condition \textbf{after} each iteration
  \item Body always executes \textbf{at least once}
  \item Great for menus and input validation
\end{itemize}
\end{frame}

% FOR LOOP
\section{for Loop}
\begin{frame}[fragile]{The \texttt{for} Loop}
\textbf{Syntax:}
\begin{lstlisting}
for (init; condition; update) {
    // body
}
\end{lstlisting}

\textbf{Example --- count from 1 to 5:}
\begin{lstlisting}
for (int i = 1; i <= 5; i++) {
    System.out.println(i);
}
\end{lstlisting}

\textbf{All three parts in one line:}
\begin{itemize}
  \item \textbf{Init}: \texttt{int i = 1} --- runs once
  \item \textbf{Condition}: \texttt{i <= 5} --- checked before each iteration
  \item \textbf{Update}: \texttt{i++} --- runs after each iteration
\end{itemize}
\end{frame}

% WHICH LOOP?
\begin{frame}{Which Loop Should I Use?}
\begin{table}
\centering
\begin{tabular}{lll}
\toprule
\textbf{Scenario} & \textbf{Loop} & \textbf{Why} \\
\midrule
Known count & \texttt{for} & Count is in the syntax \\
Unknown count, may skip & \texttt{while} & Check before running \\
Must run at least once & \texttt{do-while} & Check after running \\
Sentinel-controlled & \texttt{while} & Priming read pattern \\
Input validation & \texttt{do-while} & Must prompt at least once \\
\bottomrule
\end{tabular}
\end{table}

\vspace{0.5cm}
\textbf{Rule of thumb:} If you know how many times --- \texttt{for}. \\
If you don't --- \texttt{while} or \texttt{do-while}.
\end{frame}

% ACCUMULATORS AND COUNTERS
\begin{frame}[fragile]{Accumulators and Counters}
\textbf{Counter}: tracks how many times something happens. \\
\textbf{Accumulator}: tracks a running total.

\begin{lstlisting}
int count = 0;
double total = 0;

// inside the loop:
count++;           // counter
total += value;    // accumulator
\end{lstlisting}

\textbf{Average pattern:}
\begin{lstlisting}
double average = total / count;
\end{lstlisting}

\begin{itemize}
  \item Initialize \textbf{before} the loop
  \item Update \textbf{inside} the loop
  \item Use \textbf{after} the loop
\end{itemize}
\end{frame}

% COMMON MISTAKES
\begin{frame}[fragile]{Common Mistakes}
\textbf{1. Infinite loop} --- forgetting to update:
\begin{lstlisting}
int i = 1;
while (i <= 5) {
    System.out.println(i);
    // missing i++
}
\end{lstlisting}

\textbf{2. Off-by-one error:}
\begin{lstlisting}
// Prints 0-4, not 1-5
for (int i = 0; i < 5; i++) { ... }
\end{lstlisting}

\textbf{3. Semicolon after \texttt{for}:}
\begin{lstlisting}
for (int i = 0; i < 5; i++);  // empty loop!
{
    System.out.println(i);  // runs once
}
\end{lstlisting}
\end{frame}

% TODAY'S LAB
\begin{frame}{Today's Lab}
\textbf{Number Analyzer}
\begin{itemize}
  \item Read integers from the user until sentinel (0)
  \item Count positive and negative numbers separately
  \item Compute the running total
  \item Calculate and display the average (excluding zeros)
  \item Handle edge case: no numbers entered
\end{itemize}
\vspace{0.5cm}
\textbf{Concepts used:} \texttt{while} loop, sentinel value, accumulators, counters, conditional logic inside a loop.
\vspace{0.5cm}

\textbf{Demonstrate working code before leaving!}
\end{frame}

\end{document}
